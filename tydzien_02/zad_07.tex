\medskip
\noindent{\bf Zad. 7} 
\medskip
Mamy dostep do kul bialych oraz czarnych. Razem jest ich $b+c$, gdzie  $b$ oznacza liczbe bialych kul, $c$ - liczbe czarnych.
W zadaniu mamy wylosowac jedna kule potem ja odrzucic. Mamy wiec mozliwosc wylosowania kuli albo bialej, albo czarnej.
Rozwazmy przypadek I, gdzie wylosowalismy kule biala
  -Prawdopodobieństow wylosowania kuli bialej za pierwszym razem wynosi $b/(c+b)$
  -Przy kolejnym losowaniu, by wylosować kule biala, prawdopodobienstwo wynosi $(b-1)/(c+b-1)$
Przypadek II, wylosowano czarna
  -Prawdobienstow wylosowania za pierwszym razem czarnej $c/(c+b)$
  -Przy kolejnym losowaniu prawdobodobiestwo wylowowania bialej wynosi $b/(c+b-1)$
  
A wiec, prawdopodobienstow wydarzenia wylosowania drugiej kuli jako bialej wynosi:
\begin{center}
      $(b/(c+b))*((b-1)/(c+b-1)) + (c/(c+b))*(b/(c+b-1)) = b/(c+b)$
\end{center}
